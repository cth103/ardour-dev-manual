\documentclass[10pt,a4paper]{book}
\title{Ardour Development Manual}
\author{}

\newcommand{\todo}[1]{\textbf{XXX --- #1}}
\newcommand{\code}[1]{\texttt{#1}}
\usepackage{listings}
\lstset{frame=single,language=c++,columns=fullflexible,basicstyle=\ttfamily,commentstyle=\footnotesize\textit,texcl=true,escapechar=\@,escapebegin=\footnotesize\itshape}
\begin{document}
%\maketitle

%\chapter{Introduction}

%Welcome to Ardour!  This manual is intended to give guidance in
%Ardour's construction and how best to develop it.

%\section{Contacting other developers}

%By far the most useful means of communication with other Ardour users,
%testers and developers is via IRC.  Connect to
%\texttt{irc.freenode.org}, join the \texttt{\#ardour} channel and make
%yourself acquainted!

\chapter{Code overview}

The vast majority of Ardour is written in C++.  

Ardour's code is split into several sections, which are summarised
below by their location in the source tree.

\begin{itemize}
\item \code{gtk2\_ardour} --- the GUI front-end, written using GTK2.
  GTK is accessed using gtkmm.
\item \code{libs/ardour} --- the back-end.  This is a the UI-agnostic
  `core' of Ardour, which can be used by different user interface
  front-ends.  Currently, only the GTK2 interface is supported, but
  once upon a time a keystroke-based interface to libardour also
  existed.
\item \code{libs/pbd} --- a library of useful miscellany.
\item \code{libs/evoral} --- a library for managing lists of events of
  various types, including automation and MIDI\@.
\item \code{libs/surfaces} --- code to support hardware control
  surfaces.
\end{itemize}

\chapter{The undo system}

Ardour allows unlimited undo / redo, and saves the undo stack with its
session.  This chapter describes how this is implemented.

\section{Overview}

The undo system employs similar principles to the way in which state
is saved in sessions.  C++ object state, and changes to that state, is
described with XML\@.  This has the advantage that it can easily be
serialized to and restored from disc.

There are two main approaches to saving undo state.  In some cases,
undo is performed by taking the full state of an object before some
operation, and then the full state afterwards, and storing both as
XML\@.  Undo and redo can then be achieved simply by setting state to
from the appropriate XML node.

In some cases, however, this approach can be inefficient.  Consider,
for example, a playlist with many thousands of regions.  If we want to
represent a region's addition to the playlist, the `full state'
approach means that we would take the state of all the regions both
before and after the region add.  This in turn means that a very large
`before' and `after' XML tree is required, which is inefficient both
in terms of storage space and also computation to save and restore
state.  For cases such as this, an alternative approach is taken: only
the \emph{differences} in the `before' and `after' state are stored.
In this way, rather than using a large state containing many regions,
we consider only a small state containing information about the single
added region.  This approach is much more efficient, and the cost of
some increased implementation complexity.

Classes that must be serialized to disc, and/or those that participate
in undo, inherit from \code{PBD::Stateful}.

\subsection{Commands}

However a change in state is described, it is represented by some
subclass of \code{PBD::Command}.  A change to some object represented
by `before' and `after' state is described by a
\code{PBD::MementoCommand}.  A change represented as a difference in
state is represented by a \code{PBD::StatefulDiffCommand}.  Since some
operation that is meaningful to the user may require changes to
several different Ardour objects, commands can be grouped into a
\code{PBD::UndoTransaction}.

Consider, as an example, the movement of two regions that are on
different tracks.  To the user, this is a single `move' operation.
Internally, however, there is a change to two different playlists.
Each playlist's change is described by a \code{StatefulDiffCommand},
and these commands are collected into a single \code{UndoTransaction}
which describes the operation that the user sees.

The \code{Session} assists somewhat in creation of
\code{UndoTransaction}s by keeping a `current' one
(\code{\_current\_trans}) and providing the
\code{begin\_reversible\_command} method to create a new
\code{UndoTransaction}, the \code{add\_command} method to add a
\code{Command} to the that \code{UndoTransaction}, and then the
\code{commit\_reversible\_command} method to add the finished
\code{UndoTransaction} to the history.


\section{Full state undo / redo}



\section{Properties}

The most basic building block of the property-based undo system is the
\code{PBD::Property}.  This is a templated class which manages a
single piece of scalar state.  For example, \code{ARDOUR::Region}
contains the definition:

\begin{lstlisting}
PBD::Property<framepos_t> _position;
\end{lstlisting}

This declares a \code{Property} of type \code{framepos\_t}.  This
\code{\_position} can mostly be used as if it were a standard
variable, but the use of the \code{Property} class means that
\code{\_position} can also remember some history of previous values of
the position.  At the end of the following sequence of code:

\begin{lstlisting}
_position = 4;
_position->clear_changes ();
_position = 9;
\end{lstlisting}

the \code{\_position} property knows that its current value is 9, but
also that its previous value (before the last call to
\code{clear\_changes}) was 4.

\section{Stateful classes}

Classes such as \code{ARDOUR::Region} which have \code{PBD::Property}
members are derived from the class \code{PBD::Stateful}.  This helps
in managing the properties of its subclass.

\section{Example}

Consider a \code{Region} class which looks something like this:

\begin{lstlisting}
class Region : public Stateful {
    Property<framepos_t> _position;
    Property<framepos_t> _length;
};
\end{lstlisting}

and suppose we have an instance of this class called \code{region}
whose properties are set so that $\code{\_position} = 1$ and
$\code{\_length} = 2$.  We now want to change \code{\_position} and
\code{\_length} in an undo-able fashion.

\subsection{Making the changes}

Here's what happens when we do to make the changes:

\begin{lstlisting}
region->clear_changes ();
\end{lstlisting}

\code{Stateful::clear\_changes} calls \code{Property::clear\_changes}
on each of its properties (\code{\_position} and \code{\_length}).
This means that \code{\_position} and \code{\_length} clear their
\code{\_have\_old} flags.

\begin{lstlisting}
region->_position = 4;
region->_length = 6;
\end{lstlisting}

The properties update their current values and note their old values,
and the fact that they have been changed.

\begin{lstlisting}
session->add_command (new StatefulDiffCommand (region));
\end{lstlisting}

\chapter{Threading}

Ardour runs multiple threads.  The main ones are as follows:

\begin{itemize}
\item \textit{GUI thread} --- where the GTK main loop runs.  All
  GUI-related code \textit{must} execute in this thread.
\item \textit{Process thread} --- code which that runs the JACK
  process callback.  This is where audio processing is done.
  Operations in this thread must not perform non-realtime operations
  such as allocating memory, as these could easily cause audio
  drop-outs.
\item \textit{Metering thread} --- a thread which computes peak values
  for meters based on the current fall-off settings.
\item \textit{Peak thread}
\end{itemize}


\section{What the threads do}

\subsection{Process thread}

After some setup, the main body of the process thread is in
\code{AudioEngine::process\_callback}.  This method is called when
there audio / MIDI `work' to be done; typically it will be called when
there are $N$ frames of audio available from JACK (where $N$ is the
JACK block size).  This method takes the \textit{process lock},
\code{AudioEngine::\_process\_lock} which is held throughout the
callback.

\todo{Detail about other threads}


\section{Cross-thread signalling}

Ardour makes much use of signal / slot messaging.  At its most basic
level, this system allows some class \code{Fred} to expose some
signal `Foo'.  Another class \code{Jim} can connect to
\code{Fred::Foo}, such that when \code{Fred} emits `Foo', a method
on \code{Jim} is called.  This is the system implemented by, for
example, \code{libsigc++} as used by \code{gtkmm}.  The main
conceptual advantage of this approach is that \code{Fred} need have
no dependency on \code{Jim} at all, and furthermore that another
class \code{Sheila} may also connect to the same signal, again
without \code{Fred} being any the wiser.

Using such a system across threads (so that, for example, the process
thread can signal the GUI thread) has some complications, chief
amongst which is that signal emissions from a given thread $A$ to $B$
must not block thread $A$ while they await processing.

\todo{Other reasons for all this carry-on}

Ardour's code to handle cross-thread signalling is based on (and
wraps) the \code{boost::signals2} library.

\subsection{The \code{Signal} class}

Ardour's basic `signal' class is \code{PBD::Signal}$N$, where $N$ is
the number of parameters that the signal has.  Consider, for example,
\code{PBD::Signal0}.  If a class \code{Fred} declares such a signal

\begin{lstlisting}
class Fred {
  PBD::Signal0<void> Foo;
};
\end{lstlisting}

then other classes can connect to the signal \code{Foo} using one of
the members of \code{Signal0}.

There are two main concerns with the connection of signals.  Consider
the case where \code{Fred} emits a signal that \code{Jim} connects to.
If \code{Jim} is destroyed, it is important that \code{Fred} stops
attempting to deliver signals to it.  Also, care is needed if the code
in \code{Jim} must be executed in a different thread to the one which
emitted the signal.

\subsubsection{Connection management}

Connecting to a signal results in the creation of a `connection'
object, represented in Ardour by \code{PBD::ScopedConnection}.  A
\code{ScopedConnection} is created during connection of a signal, and
its signal is disconnected when the \code{ScopedConnection} goes out
of scope.  Ardour also provides the \code{ScopedConnectionList}, which
a class can inherit from if it connects many signals.  The detailed
use of these classes is discussed in the following sections.

\subsubsection{Connecting to the same thread}

If the connected code is to be executed in the same thread as the
signal emission, connection is simple, so that:

\begin{lstlisting}
PBD::ScopedConnection c;
foo.connect_same_thread (c, boost::bind (&Jim::bar, this));
\end{lstlisting}

or

\begin{lstlisting}

class Fred : public PBD::ScopedConnectionList {
...
};

Fred cl;

foo.connect_same_thread (cl, boost::bind (&Jim::bar, this));
\end{lstlisting}

will result in \code{Jim::bar} being called when \code{foo} is
emitted.  The \code{connect\_same\_thread} method calls the boost
library directly.

The \code{c} and \code{cl} objects in the
examples above are a \code{ScopedConnection} and a
\code{ScopedConnectionList} respectively.  Details of the new
connection are filled into the \code{ScopedConnection} or added to the
\code{ScopedConnectionList}.

The interaction of the signal and the connection are perhaps best
illustrated with some code fragments:

\begin{lstlisting}
class Foo {
public:
  void fred () {
    A ();
  }

  PBD::Signal0<void> A;
};

class Bar {
public:
  void Bar (Foo* f) {
    f->A.connect_same_thread (
        _connection, boost::bind (&Bar::sheila, this)
      );
  }

  void sheila () {
    ...
  }
};

private:
  PBD::ScopedConnection _connection;
};

Foo* f = new Foo;

// Make a new Bar and its constructor will connect to Foo
Bar* b = new Bar (f);

// Foo::fred emits A, which results in Bar::sheila being called
f->fred ();

// Deleting b means that Bar::\_connection is also deleted, which
// disconnects the connection made to Foo::A in Bar's constructor.
delete b;

// Foo::fred emits A as before, but no-one is listening!
f->fred ();
\end{lstlisting}

If \code{Bar} is larger, and connects to more signals, it may be more
convenient to modify the code above to:

\begin{lstlisting}
class Foo {
public:
  void fred () {
    A ();
  }

  PBD::Signal0<void> A;
};

class Bar : public PBD::ScopedConnectionList {
public:
  void Bar (Foo* f) {
    f->A.connect_same_thread (
        *this, boost::bind (&Bar::sheila, this)
      );
  }

  void sheila () {
    ...
  }
};
\end{lstlisting}

In this case, the connection is stored in the parent
\code{ScopedConnectionList} class, and disconnected when it is
destroyed.  This approach requires less code in many cases.

\subsubsection{Connecting to a different thread}

In one case, the emission of signals from the back-end to a user
interface (UI), it is necessary to have the signal handler running in
a different thread to the emission.  Connecting signals in this way is
quite similar to connecting same-thread signals, although the
implementation is more complicated.

Ardour has several components which it considers UI\@.  The most
obvious, perhaps, is the graphical UI; other components are the Open
Sound Control (OSC) interface, and also those provided by hardware
control surfaces.

In the Ardour code, each of these UIs inherits from \code{AbstractUI},
which in turn is-a \code{BaseUI} which is-a \code{EventLoop}.



\section{How signals get from disk to JACK outputs}

[midi]

Process thread: route / track calls diskstream::get\_playback, and then process\_output\_buffers on that.

Diskstream::get\_playback reads from its ringbuffer.


Butler thread: calls Track::do\_refill which in turn calls Diskstream::do\_refill

MidiDiskstream::do\_refill calls MidiDiskstream::read which calls MidiPlaylist::read

Assuming there's only 1 region, this calls Region::read\_at

Which reads from its source, which gets from the model, if there is one, or gets direct from the SMF.



%%%

CrossThreadPool: is-a Pool.

PerThreadPool: manages a CrossThreadPool per thread.  create\_per\_thread\_pool
called by SessionEvent::create\_per\_thread\-pool


---

The x-thread signalling problem is that we want signals from one thread to be catchable
by other threads.

In practice this means signalling from A Thread to one of the `UI' threads (GUI, OSC, MIDI).




SignalN: wrapper for boost::signals2::signal<>

ScopedConnection == boost::signals2::scoped\_connection
UnscopedConnection == boost::signals2::connection

Methods:
connect\_same\_thread: directly connect to boost signal.
connect: connects boost signal to EventLoop::call\_slot; on boost signal emission, call\_slot
does the cross-thread signalling voodoo via AbstractUI.

AbstractUI is-a BaseUI is-a EventLoop; used by OSC, MIDI, GUI

invalidation:

Connect a signal on B to A with something like

B->Signal.connect (x, invalidator (*this), boost::bind (...), gui\_context());

If A is a sigc::trackable, when it is destroyed we want to make sure that
A is not called when the signal on B happens.  The invalidator does this.


%%%


\chapter{Export}

Ardour's export facilities are comprehensive, and the code that
implements them is rather complex.  This chapter describes that export code.

\section{Capabilities}

Export has a number of parameters.  To summarise, these are:

\begin{itemize}
\item \textbf{Format}
  \begin{itemize}
  \item File format (AIFF, WAV, Ogg Vorbis etc.)
  \item Sample rate.
  \item Type of sample rate conversion, if required.
  \item Normalization
  \item Trimming of silence.
  \item Bit depth.
  \item Type of dithering, if required.
  \end{itemize}
\item \textbf{Exported file locations and names}; where to put the exported files and what to call them.
\item \textbf{Time ranges} to export.
\item \textbf{Tracks and/or busses} to export.
\end{itemize}



Export Formats

Described by subclass of ExportFormat.  Subclasses are:

ExportFormatLinear : AIFF, AU, IRCAM, WAV, W64, RAW.
ExportFormatOggVorbis
ExportFormatFLAC
ExportFormatBWF

ExportFormatBase 

This is:
 format id (F\_WAV, F\_W64, F\_AIFF, F\_AU, F\_IRCAM, F\_RAW, F\_FLAC, F\_Ogg)
 quality (LosslessLinear, LosslessCompression, LossyCompression)
 sample format (8-bit, 16-bit, 24-bit, 32-bit, U8, float, double, Vorbis)



\end{document}
